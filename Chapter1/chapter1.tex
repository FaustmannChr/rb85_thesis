%!TEX root = ../thesis.tex
% Spellchecker ignore
% cSpell:ignore AOMs, depump, acousto, levelscheme, Oshea, retroreflection, antinode, itemsep, redbluedetuning, trapdepth, giga, detuningopt, cmmnt, waals, potentialoverlap, eqref, dipolepot, wrapfigure, intensityplot, includegraphics, retroreflected, detunings, linewidth, dispersive, antinodes, pagebreak, ifpdf, graphicspath, detuned, bigskip, nano, citep, milli, detuning, medskip, mathrm
%*******************************************************************************
%****************************** First Chapter **********************************
%*******************************************************************************

\chapter{\label{chap:dipole}Theory of laser trapping of atoms}


\ifpdf{}
    \graphicspath{{Chapter1/Figs/Raster/}{Chapter1/Figs/PDF/}{Chapter1/Figs/}}
\else
    \graphicspath{{Chapter1/Figs/Vector/}{Chapter1/Figs/}}
\fi

Atoms can be trapped in an optical potential created by the dispersive interaction 
of the atomic dipole moment with the intensity gradient of the light field. These 
trap can be used to confine the atoms in 3D counteract gravity. In the case of 
large detunings the expressions for the dipole potential and scattering rate\cite{grimm} 
are the following:
%
\begin{align}
    U_\mathrm{dip}(z) =& \frac{3\pi c^2}{2\omega_0^3}~\frac{\Gamma}{\Delta}~I(z)~, \label{eq:dipolepot_simple}\\
    \Gamma_{\mathrm{sc}}(z) =& \frac{3\pi c^2}{2\hbar\omega_0^3}~
    {\left ( \frac{\Gamma}{\Delta} \right )}^2~I(z)~.\label{eq:scattering_simple}
\end{align}

\begin{wrapfigure}{R}{.4\textwidth}
    \centering
    \includegraphics[width=0.35\textwidth]{redbluedetuning}    
    \caption{\label{fig:redbluedetuning} Illustration of dipole traps with red 
    and blue detuning. The grey area represents regions of high intensity.}
\end{wrapfigure}
Where \textit{c} is the speed of light, \(\Gamma \) is the coupling strength
between the two atomic levels of the atomic transition, \(\Delta \) is the detuning 
between the light and the atomic transition (\(\Delta := \omega - \omega_0 \)), 
where \(\omega \) is the driving pulsation of the light field and \(\omega_0 \) 
the atomic transition pulsation \(\omega_0=2\pi~\frac{c}{\lambda_0}=2\pi\nu_0 \) 
and \(I(z) \) corresponds to the intensity of the light field at a distance from 
the resonator. Dipole traps can be divided into two main classes, red-detuned 
traps (\(\Delta < 0 \)) and blue-detuned traps (\(\Delta > 0 \)). Below an atomic 
resonance (red) the dipole potential is negative and the potential minima are 
therefore found at positions of maximum intensity. For a blue-detuned light the 
potential minima correspond to minima of the intensity and the interaction repels atoms from the field as seen in Fig.
~\ref{fig:redbluedetuning}. For our setup we will consider a red-detuned trap.
\pagebreak

There are three different trap configurations possible:
\begin{itemize}
    \setlength{\itemsep}{0ex}
    \item \textit{focused-beam trap}: Focused-beam traps are consisting of a 
    single strongly focused beam and have a confinement volume proportional to 
    \(z_R \times {w_0}^2 \), where \(z_R \) is the Rayleigh length and \(w_0 \) 
    is the beam radius.
    \item \textit{crossed-beam trap}: A crossed-beam configuration uses two or 
    more beams which intersect at their foci. The confinement volume reduces to 
    the intersection of the contributing beams, which is in case of \(w_0 \) 
    smaller than \(w_1 \) the cross section of ``beam 0'' times the diameter of 
    ``beam 1''.
    \item \textit{standing wave trap}: In case of a standing wave trap the atoms 
    are axially confined in the antinodes of a standing wave. If the standing 
    wave is created through a retroreflection the first antinode and therefore 
    first trapping site lies at a distance of \(\frac{\lambda}{4} \) from the 
    surface (see Fig.~\ref{fig:intensityplot}). The resulting volume is in the 
    order of \({w_0}^2 \times \frac{\lambda}{4} \).
\end{itemize} 

For our Experiment it is important that the atoms are as close as possible from the
resonator, because the interacting evanescent field of the photon in the resonator
decreases exponentially with a decay length of \(\frac{\lambda_{at} }{2\pi} \) where 
\(\lambda_{at} \) is the wavelength of the cavity field tuned to the atomic 
transition. \(5S_{1/2} \rightarrow 5P_{3/2} \) (\(D_2 \) line of Rb at 
\SI{780}{\nano\meter}) has a decay length of \SI{124}{\nano\meter}.

\begin{wrapfigure}{R}{.4\textwidth}
    \centering
    \includegraphics[width=0.35\textwidth]{intensityplot}    
    \caption{\label{fig:intensityplot} 1D Intensity distribution of a retroreflected 
    gaussian laser beam for a partial standing wave with \(r=-0.2\).}
\end{wrapfigure}
If one uses a laser closely detuned from 
the \(D_2 \) line \(5S_{1/2} \rightarrow 5P_{3/2} \), i.e. \SI{780}{\nano\meter}, 
this leads to a distance of \(\frac{\lambda}{4} \approx \SI{190}{\nano\meter} \).
This would be to far from the resonator which is already too far. Rubidium atoms 
also have a transition from \(5S_{1/2}~\text{to}~6P_{3/2} \) at \SI{420}{\nano\meter}.
The transition strength are much weaker for this line, and the aim of this thesis 
is to decide on the feasibility of such a trap and as the case may be determine the
parameters for it.
\pagebreak

\begin{figure}[h]
    \centering
    \includegraphics[width=0.9\textwidth]{levelscheme}
    \caption{\label{fig:levelscheme} Level scheme of \(^{85}\)Rb with lifetimes
    and transition wavelength of the first states.}
\end{figure}

As we can see in Eq.~\eqref{eq:dipolepot_simple} the dipole potential is proportional 
to \(\frac{I}{\Delta} \), which means that we have two degrees of freedom, the 
detuning and the power (\(I\propto\frac{P}{cross section} \)). To choose our 
parameters properly we have to consider some constraints. We want to keep the 
scattering rate as low as possible, because each photon scattered may depolarize 
the atom or also lead to depump the atom in which case it cannot be detected
anymore. The scattering rate is proportional 
to \(\frac{I}{\Delta^2} \) (Eq.~\ref{eq:scattering_simple}), so a detuning as big 
as possible would be beneficial. On the other hand laser power is limited 
(\(P_{\max}=\SI{80}{\milli\watt} \)), due to the maximum power output of our laser.
And additionally we do not want to send too much power onto the resonator, because 
due to the highly focused beam (\(w_0 = \SI{3.6}{\micro\meter} \)) and the more
energetic wavelength the resonator could be damaged. To compensate the lower laser 
power the detuning has to be small. However if \(\Delta \) smaller than the separation 
between the \(D_1 \) and \(D_2 \) lines (\SI{420} \& \SI{421}{\nano\meter}) then 
we need to take the fine structure of the atom into account and equations
~\eqref{eq:dipolepot_simple} and~\eqref{eq:scattering_simple} rewrite as 
%
\begin{align}
    U_\mathrm{dip}(z) =& \frac{\pi c^2}{2\omega_0^3}~\left( 
        \frac{2~\Gamma_{\omega,D2}}{\Delta_{D2}} + 
        \frac{\Gamma_{\omega,D1}}{\Delta_{D1}} \right)~I(z)~, \\
    \Gamma_{\mathrm{sc}}(z) =& \frac{\pi c^2}{2\hbar\omega_0^3}~\left( 
        \frac{2~\Gamma_{\omega,D2}~\Gamma_{\omega,D2,tot} }{\Delta_{D2}^2 } + 
        \frac{\Gamma_{\omega,D1}~\Gamma_{\omega,D1,tot} }{\Delta_{D1}^2 } \right)~I(z)~.
\end{align}
%
% TODO: Add explanation of the factor of 2 in above equation and why in the 
%original equation is only one Gamma

It should be noted that as \(6P_{1/2}~\text{and}~6P_{3/2} \) are not the first
excited states, there exist several possible decay channels from there to the
ground state. \(\Gamma_{\omega,Dx} \) are the transition strength from 
\(5S_{1/2} \rightarrow 6P_{1/2} \) and \(5S_{1/2} \rightarrow 6P_{3/2} \), while
\(\Gamma_{\omega,Dx,tot} \) are total decay rates of the \(6P_{1/2}~\text{and}~6P_{3/2} \)
states with \(\frac{1}{\Gamma_{\omega,Dx,tot}} \) the mean lifetime of the state 
and \(\Delta_{Dx} \) represents \(\omega - \omega_{0,Dx} \). All values can be 
found in table~\ref{table:iso_prop}. In our case we are red-detuned from the 
\(D_2 \)-transition and the detuning will be smaller than the fine structure 
splitting of \SI{2.32}{\tera\hertz}, so we get an additional counter term of the 
\(D_1 \)-transition (blue-detuned).

It should be noted that \(\Gamma_{6P_{3/2}}\) is 20 times weaker than 
\(\Gamma_{5P_{3/2}}\) and thus, for the same trap depth a shorter detuning or 
higher power will be needed comparatively to a dipole trap around \SI{780}{\nano\meter}.

\subsection*{Possible configuration of optical dipole trap}
\begin{figure}[h]
    \centering
    \includegraphics[width=0.6\textwidth]{resonator_trap_label}
    \caption{\label{fig:resonator_trap_label} Experimental setup and intensity distribution. }
\end{figure}
To calculate the trap potential we have to derive the intensity of the standing
wave. The trap is produced by a beam focused down to at radius of \(w_0 = 
\SI{3.6}{\micro\meter} \). While the resonator diameter is \SI{36}{\micro\meter}
(as shown in Fig.~\ref{fig:resonator_trap_label}). Thus, one can consider that
the beam is retroreflected on a planar surface. With this assumption the electric 
field becomes:
\begin{align}
    \vec{E}(z) = E_0~\vec{x}~\left( e^{-i k_z z} + r e^{i k_z z} \right)
\end{align}
with \(r \) as the reflection coefficient in amplitude, which is in case of a 
glass/vacuum interface \(r = -0.2 \). \(k_z \) is the wave vector in z-direction 
and equal to \(\frac{2\pi}{\lambda} \). This leads to the intensity
\begin{align}
    I(z) = \frac{ { 2| \vec{E}(z) | }^2}{\eta}
\end{align}
depicted in Fig.~\ref{fig:intensityplot}. Where \(\eta \) is the characteristic 
impedance and \(\frac{2~{E_0}^2}{\eta} \) will be denoted as the maximum intensity 
\(I_0 \). Related to the power of a gaussian beam as
\begin{align}
    P_0 = \frac{1}{2}~I_0 {w_0}^2 \pi~.
\end{align} 
The range of power accessible is between \(0~<~P_0~<\SI{80}{\milli\watt} \), due
the maximum laser output power. 

\begin{wrapfigure}{Hr}{.4\textwidth}
    \includegraphics[width=0.35\textwidth]{potentialoverlap}    
    \caption{\label{fig:potentialoverlap} Overlap of dipole and Van-der-Waals potential.}
    \vspace{2em}
    \includegraphics[width=0.35\textwidth]{detuningopt}    
    \caption{\label{fig:detuningopt} Trap depth for different detuning.}
\end{wrapfigure}

The trap depth must be bigger than the energy of the atom, here the kinetic energy 
\(E_{kin} = \frac{1}{2}m_{Rb} v^2 \) due to the fact that the atoms are free 
falling up to \SI{60}{\milli\second}. It corresponds in terms
of temperature to \(E_{kin}/k_B = \SI{1.77}{\milli\kelvin} \). Our trap is 
conservative. When an atom is captured it gains the potential energy at its
position \(E_{pot}(z) \) and if \(E_{kin} + E_{pot}(z) \) less than \(E_{pot,\max}\) 
then the atom will not be trapped. Therefore one should add a safety margin to 
capture more atoms. For example a \SI{5}{\milli\kelvin} trap would capture 
\(\frac{5-1.77}{5} \approx \SI{60}{\percent} \) of atoms entering the trap in the 
worst case. In our setup we are so close to the resonator that the atoms are also 
sensitive to the Van-der-Waals potential\cite{PhysRevA.89.022511}
\begin{align}
    U_{VdW} = -\frac{C_3}{r^3}
\end{align} 

(see Fig.~\ref{fig:potentialoverlap}). The \(C_3\)
coefficient is \( h\cdot770\cdot10^{-18} \si{\hertz\meter\cubed}\)~\cite{OsheaPHD}, 
where \(h\) is Plank's constant.

\begin{wrapfigure}{Hr}{.4\textwidth}
    \includegraphics[width=0.35\textwidth]{trapdepth}    
    \caption{\label{fig:trapdepth} Calculated trap potential for \SI{20}{\milli\watt}
     power and a \SI{6}{\giga\hertz} detuning.}
\end{wrapfigure}

The total potential seen by the atoms is \(U_{VdW} + U_{dip} \). The depth of the 
potential well is now the difference between the reduced first maximum and the 
first minimum at \(\lambda/4 \). For our trap we expect to have \SI{20}{\milli\watt} 
of laser power accessible. The power losses include locking the laser to a fixed
frequency (\SI{5}{\milli\watt}), because the frequency deviation in free running 
can be on the order of \SI{1}{\giga\hertz} when we need to be only \SI{10}{\giga\hertz}
from resonance away (see. Fig.~\ref{fig:detuningopt}). One also needs to take into 
account the losses due to the optical path of the beam before reaching the atoms,
which will contain acousto-optical modulators (AOMs) and fiber coupling. As we 
can see in Fig.~\ref{fig:detuningopt} for \SI{20}{\milli\watt} power the detuning 
has to be lower than \SI{7}{\giga\hertz}. For a detuning of \SI{6}{\giga\hertz} 
we get a trap depth of \SI{5.5}{\milli\kelvin} as shown in Fig.~\ref{fig:trapdepth}.

In order to set the laser detuning properly, we need to realize a spectroscopy 
on the \(5S_{1/2} \rightarrow 6P_{3/2} \) transition. (see Chapter 3)

As seen above in all the formulas to calculate the trap depth the transition
strength plays a significant role. To be sure we are using the right value we will
measure it in our lab. For that we will measure the related intensity saturation
of the \(5S_{1/2} \rightarrow 6P_{3/2} \) transition. It marks the point where 
excited state atoms are equally likely to decay by stimulated emission or by 
spontaneous emission. Afterwards we can check with a theoretical relation where 
every other value is known, if the value is correct.
\begin{align}
    I_{s,420} &= \frac{\Gamma_{\omega,tot,420}\cdot{\omega_{420}}^3\cdot I_{s,780}}
    {\Gamma_{\omega,420}\cdot\Gamma_{\omega,780}\cdot{\omega_{780}}^3}
\end{align}  
The index 420 corresponds to \(5S_{1/2} \rightarrow 6P_{3/2} \) and the index 780 
to \(5S_{1/2} \rightarrow 5P_{3/2} \). (see Chapter 4)
In the next chapter we will discuss how atoms affect laser photon in order to 
perform a absorption spectroscopy.

