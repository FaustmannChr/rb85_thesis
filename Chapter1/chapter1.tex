%!TEX root = ../thesis.tex
%*******************************************************************************
%****************************** First Chapter **********************************
%*******************************************************************************

\chapter{Theory of laser trapping of atoms}

\ifpdf{}
    \graphicspath{{Chapter1/Figs/Raster/}{Chapter1/Figs/PDF/}{Chapter1/Figs/}}
\else
    \graphicspath{{Chapter1/Figs/Vector/}{Chapter1/Figs/}}
\fi

The strategy pursued to trap is a optical dipole trap. For that the laser light 
needs to be detuned from a resonance of the atom. Thereafter the atoms are trapped
to the maxima of intensity for a red detuned laser. The beam will be reflected 
from the resonator surface and creates thereby a standing wave. The 1\(^{st}\)
maxima is at \(\sfrac{ \lambda_{trap}}{4}\).
\bigskip

So how to choose \(\lambda_{trap}\)?
\bigskip

Because of the interaction with the BMR evanescent field the atoms need to
be trapped really close: \(\frac{\lambda}{2\pi}\approx \SI{130}{\nano\meter}\) \\
Most common resonance of rubidium is \(5S_{1/2} \rightarrow 5P_{3/2}\) @
\SI{780.24}{\nano\meter}. If we use a laser red-detuned from \(\lambda \) = 
\SI{780.24}{\nano\meter} then our first maxima would be at \SI{195}{\nano\meter}
\(\Rightarrow \) Not close enough!
\bigskip

But rubidium has another transition from \(5S_{1/2} \rightarrow 6P_{3/2}\) @
\SI{420.29}{\nano\meter}, which leads to a distance of \SI{105}{\nano\meter} from
the BMR to the 1\(^{st}\) maxima. But in the formlula \citep{grimm} of the trap potenial (\(U_{dip}\))
arises the transition strength (\(\Gamma \)) of this specific transition:
\bigskip
\begin{align*}
    U_{dip}(\mathbf{r})=-\frac{\pi c^2}{\hbar\omega_0^3} \left( \frac{\Gamma}{\Delta} \right) I(\mathbf{r})
\end{align*}
\bigskip
We have to compare this potential to the kinetic energy of our rubidium atoms. The atoms fall approximately
\SI{60}{\milli\second} and the corresponding kinetic engergy would be \(E_{kin} = \frac{1}{2} m_{Rb} v^2\).
In terms of temperature we would get: \(\frac{E_{kin}}{k_B} = \SI{1.77}{\milli\kelvin}\). This is quite huge
for a dipole trap. For that reason one needs to know \(\Gamma_{420nm-Line}\) and one also needs to have a trap
with a small detuning. This requires to see the transitions to have a reference to lock the laser afterwards.
\bigskip
To determine \(\Gamma_{420nm-Line}\) we can use the relation with the intensity saturation:
\begin{align*}
    I_{sat,420} &= \frac{\Gamma_{tot,420}\times\omega_{420}^3\times I_{sat,780}}{\Gamma_{420}\times\Gamma_{780}\times\omega_{780}^3} \\ \\
    \text{with~~~} \Gamma_{tot,420} &= \frac{1}{total~lifetime~of~6P_{3/2}~state}
\end{align*}

\(\Rightarrow \) We want to measure \(I_{sat}\) for the blue \SI{420.29}{\nano\meter}-line. 

