%!TEX root = ../thesis.tex
% Spellcheck ignore
% cSpell:ignore addcontentsline, freerunning, detuning, microresonator, misestimates, Mathematica, modelfit, kappaplot, isat, powermeter, mathrm, giga, kappamodelratio, realrubidium, hyperfine, kappacorrection, intcorrection, powercorrection, multicolumn, extracolsep, toprule, midrule, bottomrule, fitplot, gaussians, nonumber, spectrumlegend, FWHM, diaminplot, meanplot, diaplot, bigskip, captionof, minipage, pagebreak, wincam, includegraphics,wrapfigure, ifpdf, graphicspath
%*******************************************************************************
%******************************* Fifth Chapter *********************************
%*******************************************************************************

%\chapter{Comparison with theory and conclusion}
%\chapter{Conclusion}
\chapter*{Conclusion} 
\addcontentsline{toc}{chapter}{Conclusion}

% **************************** Define Graphics Path ****************************
\ifpdf{}
    \graphicspath{{Chapter5/Figs/Raster/}{Chapter5/Figs/PDF/}{Chapter5/Figs/}}
\else
    \graphicspath{{Chapter5/Figs/Vector/}{Chapter5/Figs/}}
\fi

The values of \(I_{s}\) for the different fits are quiet scattered. Normally
one should measure the same value for both isotopes and independently of the
internal state. This is an indication that we probably
underestimated error bars. For example, we considered a scan over the line to be
linear in frequency but it is probably a sinusoidal modulation and we did not
take that into account in your fits. Additionally the obtained value is 
much smaller than the theoretical value (\SI{126}{\watt\per\meter\squared}).
This is not what we would expect. Due to deficiencies, like a finite transmission
of the cell windows, normally one has to send more power than in theory. The 
discrepancy of a factor of 4 between the measurement and the theory can still be
understood by different misestimates during the measurement and data processing. \\
One reason would be that the Beer-Lambert law was not valid, because our beam 
intensity was to high. This was checked by selecting only measurements with lower
intensity, but it only leads to a \SI{10}{\percent} improvement. 
Another reason for the deviation could be the use of the wrong polarization for
the beam. Instead of \(\pi \)-polarization it should be \(\sigma_{+}\) or 
\(\sigma_{-}\) to drive a closed transition.  
But the most prominent cause for our low saturation intensity is by far the rough 
shape of our beam profile, which leads to an inaccurate determination of the input 
intensity. Due to the power fluctuations of \SI{30}{\percent} over the beam diameter, 
part of the atoms would saturate before the rest. One could now calculate a more
complicated intensity profile and repeat the analysis. An overall increase in 
intensity would improve \(I_{s}\) in direct proportions.\\
In conclusion, in this thesis we have studied the possibility to use the \(5S_{1/2}\)
to \(6P_{3/2}\) transition to create a dipole trap for atoms close to a microresonator
surface. We have then reported on the implementation of a new laser bench used
to realize this dipole trap. And in order to confirm that the order of magnitude
of the transition strength \(\Gamma_{6P_{3/2}}\) was indeed correct, we performed
a spectroscopy measurement, leading to the evaluation of \(I_{s}\) for this
transition. This spectroscopy could also be used to lock the laser on an atomic
resonance, which will be needed, since the estimated detuning needed for the trap
to be deep enough is to small to let the laser freerunning. 

