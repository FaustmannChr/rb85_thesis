%!TEX root = ../thesis.tex
%*******************************************************************************
%****************************** Introduction ***********************************
%*******************************************************************************

\chapter*{Introduction}

% **************************** Define Graphics Path **************************
\ifpdf{}
    \graphicspath{{Introduction/Figs/Raster/}{Introduction/Figs/PDF/}{Introduction/Figs/}}
\else
    \graphicspath{{Introduction/Figs/Vector/}{Introduction/Figs/}}
\fi

One of Prof.\ Rauschenbeutel projects uses a novel type of whispering-gallery-mode (WGM)
resonator interfaced via nanowaveguides and coupled to single Rubidium atoms to carry out
experiments in the realm of Cavity Quantum Electrodynamics. The WGM resonator
is a so-called bottle-microresonator (BMR) manufactured from a standard optical glass
fiber in a heat and pull process. The light is radially confined inside the resonator by
total internal reflection and propagates along the circumference of the resonator. In such
a structure, a signicant fraction of the light field propagates in the evanescent field. By
overlapping this field with the evanescent field of an optical nanofiber, light can be coupled
into and out of the resonator very efficiently. Due to the extremely low absorption
of silica (and low surface roughness) we can produce bottle-resonators with ultra-high
optical Q-factor exceeding 10\(^{8}\). Rubidium atoms are delivered to the resonator using
an atomic fountain. For the moment the atoms are only flying by the resonator and when they enter 
the evanescent field of the BMR, they are coupled to the cavity light field. But only for
\textasciitilde{}\SI{2}{\micro\second} and moreover the distance between the resonator and the
atom is not controlled. This prevents the realization from more complicated experiments.
For that reason one needs to trap the atom.
